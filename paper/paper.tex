
\title{ Data Encryption Standard Implementation 4003-482-01 RIT }

\author{
Brian Gianforcaro \and Sam Milton \\
}

\date{\today}

\documentclass[12pt]{article}

\usepackage{graphicx}
\usepackage{multicol}

\usepackage{acronym}
 \usepackage[bookmarks=false,colorlinks=false,linkcolor={black},pdfstartview={XYZ null null 1.22}]{hyperref}
\usepackage{url}

\acrodef{DES}[DES]{Data Encryption Standard}
\acrodef{TDD}[TDD]{Test Driven Development}
\acrodef{OOP}[OOP]{Object Oriented Programming}

\begin{document}

\maketitle

\begin{abstract}
  \ac{DES} is the 56-bit block cipher that was once used to encrypt all US Government non-classified documents.
\end{abstract}

\newpage
\section{The Cryptographic Primitive}

\section{Examples}
\section{Software Design}

We chose to implement DES completely in $C++$ for this project.
Both of the members our group were knowledgeable and comfortable
with $C++$ and in general the $C$ world. We felt we would have a more
generally usable and useful implementation if we went with $C++$ over
another language (Java, Python, ... ). We attempted to take advantage ac{OOP} but
were careful not to over do it and dive into downword spiral subclass's and
analysing of class cohesion. The $DES$ implementation is simply a class, with actions of encrypting and
decrypting. All state information is organized within the class. We then
implemented a $BlockMode$ class which implements a cipher block, which is
as you know, just a wrapper around $DES$. The user is then provided with
two small utility functions $encrypt$ \and $decrypt$.

The general ideology of this first pass is that we want the code to be clean and correct.
We are not putting too much thought into optimization and efficiency. We are very familiar
with the old Donal Knuth quote:


   "We should forget about small efficiencies, say about 97\% of the time: premature optimization is the root of all evil"


During the duration of the project we attempted to work in a \ac{TDD} model.
All major functions and actions have associated unit-tests which we derived from various peices of
literature on DES. These unit test's were then hooked up to run using the Google Test Framework\cite{gtest}.

\section{Original Run time Measurements}
\section{Original Run time Analysis}
\section{Revised Software design}
\section{Revised Run time Measurements}
\section{Revised Run time Analysis}
\section{Developers manual}
\section{Users Manual}

\begin{verbatim}

  Usage: ./encrypt_opt <infile> <outfile> <key>

    <infile>  - The file you wish to encrypt
    <outfile> - The resulting encrypted file
    <key>     - 8 character string for your key

  OR

  Usage: ./encrypt_orig <infile> <outfile> <key>

    <infile>  - The file you wish to encrypt
    <outfile> - The resulting encrypted file
    <key>     - 8 character string for your key


\end{verbatim}

\begin{verbatim}

  Usage: ./decrypt_opt <infile> <outfile> <key>

    <infile>  - The file you wish to decrypt
    <outfile> - The resulting decrypted file
    <key>     - 8 character string for your key

  OR

  Usage: ./decrypt_orig <infile> <outfile> <key>

    <infile>  - The file you wish to decrypt
    <outfile> - The resulting decrypted file
    <key>     - 8 character string for your key

\end{verbatim}

\section{What we learned}
\section{Future Work}
\section{Team Member Work}
\section{References}

\begin{thebibliography}{99}
  \bibitem{gtest}{ \href{ http://code.google.com/p/googletest/ }{Google Unit Testing Framework}}
\end{thebibliography}

\end{document}
